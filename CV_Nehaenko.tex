%----------------------------------------------------------------------------------------
%	DOCUMENT DEFINITION
%----------------------------------------------------------------------------------------

\documentclass[a4paper,12pt]{article}
\usepackage{enumitem}
\setlist{nolistsep}
\usepackage[T2A]{fontenc}
\usepackage[utf8]{inputenc}
%----------------------------------------------------------------------------------------
%	FONT
%----------------------------------------------------------------------------------------

% % fontspec allows you to use TTF/OTF fonts directly
% \usepackage{fontspec}
% \defaultfontfeatures{Ligatures=TeX}

% % modified for ShareLaTeX use
% \setmainfont[
% SmallCapsFont = Fontin-SmallCaps.otf,
% BoldFont = Fontin-Bold.otf,
% ItalicFont = Fontin-Italic.otf
% ]
% {Fontin.otf}

%----------------------------------------------------------------------------------------
%	PACKAGES
%----------------------------------------------------------------------------------------
\usepackage{url}
\usepackage{parskip} 	

%other packages for formatting
\RequirePackage{color}
\RequirePackage{graphicx}
\usepackage[usenames,dvipsnames]{xcolor}
\usepackage[scale=0.9]{geometry}

%tabularx environment
\usepackage{tabularx}
\usepackage{mathtools}

%for lists within experience section
\usepackage{enumitem}

% centered version of 'X' col. type
\newcolumntype{C}{>{\centering\arraybackslash}X} 

%to prevent spillover of tabular into next pages
\usepackage{supertabular}
\usepackage{tabularx}
\newlength{\fullcollw}
\setlength{\fullcollw}{0.47\textwidth}

%custom \section
\usepackage{titlesec}				
\usepackage{multicol}
\usepackage{multirow}

%CV Sections inspired by: 
%http://stefano.italians.nl/archives/26
\titleformat{\section}{\large\scshape\raggedright}{}{0em}{}[\titlerule]
\titlespacing{\section}{0pt}{6pt}{6pt}

%for publications
\usepackage[style=authoryear,sorting=ynt, maxbibnames=2]{biblatex}

%Setup hyperref package, and colours for links
\usepackage[unicode, draft=false]{hyperref}
\definecolor{linkcolour}{rgb}{0,0.2,0.6}
\hypersetup{colorlinks,breaklinks,urlcolor=linkcolour,linkcolor=linkcolour}
\addbibresource{citations.bib}
\setlength\bibitemsep{1em}

%for social icons
\usepackage{fontawesome5}

%debug page outer frames
%\usepackage{showframe}

%----------------------------------------------------------------------------------------
%	BEGIN DOCUMENT
%----------------------------------------------------------------------------------------
\begin{document}

% non-numbered pages
\pagestyle{empty} 

%----------------------------------------------------------------------------------------
%	TITLE
%----------------------------------------------------------------------------------------

% \begin{tabularx}{\linewidth}{ @{}X X@{} }
% \huge{Your Name}\vspace{2pt} & \hfill \emoji{incoming-envelope} email@email.com \\
% \raisebox{-0.05\height}\faGithub\ username \ | \
% \raisebox{-0.00\height}\faLinkedin\ username \ | \ \raisebox{-0.05\height}\faGlobe \ mysite.com  & \hfill \emoji{calling} number
% \end{tabularx}

\begin{tabularx}{\linewidth}{@{} C @{}}
\huge{Павел Нехаенко} \\[4.5pt]
\normalsize{Ярославль, Россия} \\[6.5pt]
\href{https://github.com/sampleeer}{\raisebox{-0.05\height}\faGithub\ Github} \ $|$ \ 
\href{https://www.linkedin.com/in/pavel-nekhaenko-499496254}{\raisebox{-0.05\height}\faLinkedin\ LinkedIn} \ $|$ \ 
\href{https://pashaprofile.neocities.org}{\raisebox{-0.05\height}\faGlobe \ Profile} \ $|$ \ 
\href{mailto:p.nekhaenko@yandex.ru}{\raisebox{-0.05\height}\faEnvelope \ p.nekhaenko@yandex.ru} \ $|$ \ 
\href{tel:+000000000000}{\raisebox{-0.05\height}\faMobile \ +7(920)650-79-43} \\
\end{tabularx}

%----------------------------------------------------------------------------------------
% EXPERIENCE SECTIONS
%----------------------------------------------------------------------------------------

%Interests/ Keywords/ Summary
\section{Образование}
\textbf{\small Яргу им. П.Г. Демидова}
\textit{\footnotesize Бакалавриат}& \hfill \small Сентябрь 2021 - Июнь 2025 \\ 
\textit{ \footnotesize Прикладная математика и информатика} 

%Experience
\section{Опыт}

\begin{tabularx}{\linewidth}{ @{}l r@{} }
\textbf {\small СБЕР} 
\textit{\footnotesize Data Scientist} & \hfill \footnotesize  Июль 2024 -  \\[1.75pt]
\multicolumn{2}{@{}X@{}}{
\begin{minipage}[t]{\linewidth}
    \begin{itemize}
    \setlength\itemsep{0.5em}
        \item \small Разработка алгоритма анализа временных рядов для прогнозирования длительности стадий сделок с использованием тестов ADF и моделей авторегрессии, повысив точность прогнозов на 12 \%.
        \item \small Построение ETL-пайплайнов, используя Apache Spark и Greenplum.
        \item \small Проведение очистки и предобработки данных для последующего обучения.
    \end{itemize}
    \end{minipage}
}
\end{tabularx}




%Projects
\section{Проекты}

\begin{tabularx}{\linewidth}{  @{}l r@{} }
 \textbf {\small \href{https://github.com/Lambda-Developement}{Parking Map Application}} & \textit{\footnotesize $\mid$ PHP, Java Script, GLSL, HTML, CSS, Python, Apache Cordova} 
 \hfill \small  \textcolor{white}{May 2024} \\[1.75pt]

\multicolumn{2}{@{}X@{}}{
\begin{minipage}[t]{\linewidth}
    \begin{itemize}
    \setlength\itemsep{0.5em}
        \item \small Создали мобильное приложение в команде, используя адаптивную верстку, которая была преобразована в Apache Cordova.
        \item \small Обучили нейронную сеть AlexNet определять свободные и занятые места.
        \item \small Нейронная сеть была написана на Keras для анализа видео в реальном времени с видеокамер в городе для трансляции бесплатных парковочных мест.
    \end{itemize}
    \end{minipage}
}  \\
\end{tabularx}




\begin{tabularx}{\linewidth}{ @{}l r@{} }
\textbf {\small \href{https://github.com/sampleeer/POW-Digital_breakthrough_2022}{AI Walrus Recognizer}} & \textit{\footnotesize $\mid$ Python, Jupyter Notebook, LiteSQL, Flutter} 
\hfill \small  \textcolor{white}{2025} \\[1.75pt]
\multicolumn{2}{@{}X@{}}{
\begin{minipage}[t]{\linewidth}
    \begin{itemize}
    \setlength\itemsep{0.5em}
        \item \smallСоздали кроссплатформенное приложение для обработки фотографий.
        \item \small Обучили нейронную сеть YOLOv5 на наборе данных для подсчета количества моржей на фотографии.
    \end{itemize}
    \end{minipage}
}  \\
\end{tabularx}

%----------------------------------------------------------------------------------------
%	EDUCATION
%----------------------------------------------------------------------------------------
\section{Награды и победы}
\begin{tabularx}{\linewidth}{ @{}l r@{} }
\textbf{\small Математическая олимпиада Яргу}\\[1.75pt] 
\textit{\footnotesize \href{https://math.uniyar.ac.ru/news/winner-2020-dean}{Math.uniyar.ru.winner}}  \\[1.75pt]
\multicolumn{2}{@{}X@{}}{
\begin{minipage}[t]{\linewidth}
    \begin{itemize}
    \setlength\itemsep{0.5em}
        \item \small Был награжден дипломом второй степени, который позволил мне учиться в университете по углубленной программе CIS (Center of Integrable Systems).
    \end{itemize}
    \end{minipage}
}  \\
\end{tabularx}

\begin{tabularx}{\linewidth}{ @{}l r@{} }
\textbf{\small Leaders of Digital Hackathon ’’Autonomous logistics” победитель}  \\[1.75pt] 
\textit{\footnotesize \href{https://drive.google.com/file/d/1Ph7rNf6mNehsl9VWehzMnmgnoe7ciOXl/view?usp=sharing}{Leadersofdigital.ru}}  \\[1.75pt]
\multicolumn{2}{@{}X@{}}{
\begin{minipage}[t]{\linewidth}
    \begin{itemize}
    \setlength\itemsep{0.5em}
        \item \small Создали приложение для города Москвы для анализа пробок на дорогах с объединением Google Maps, с внедрением геймификации для удобства пользователей, которое победило среди более чем 300 других участников.
    \end{itemize}
    \end{minipage}
}  \\
\end{tabularx}

\begin{tabularx}{\linewidth}{ @{}l r@{} }
\textbf{\small Digit Hackathon and Acceleration Program ’’Agronomy” победитель} \\[1.75pt] 
\textit{\footnotesize \href{https://drive.google.com/file/d/1V5Pg9bvlUpusp6y3rz6BZiyhBOJCoFhx/view?usp=sharing}{Leader-id.ru}}  \\[1.75pt]
\multicolumn{2}{@{}X@{}}{
\begin{minipage}[t]{\linewidth}
    \begin{itemize}
    \setlength\itemsep{0.5em}
        \item \small Создали кроссплатформенное приложение, которое позволяет людям создавать виртуальную разметку для домашнего скота, отслеживать геолокацию онлайн и управлять перемещением животных. Также была представлена 3D-модель разработанных ошейников с использованием Arduino и программирования в сети LoRaWAN.
        \item \small Было получено финансирование для реализации проекта, который ранее был создан на хакатоне.
    \end{itemize}
    \end{minipage}
}  \\
\end{tabularx}


%----------------------------------------------------------------------------------------
%	SKILLS
%----------------------------------------------------------------------------------------
\section{Технические навыки}
\textbf {\small Языки:} & \textmd {\small Python, C++, Mathematica, LaTEX}\\[1.75pt]
%\textbf {\small Developer Tools:} & \textmd {\small VS Code, Jupyter Notebook}\\[1.75pt]
\textbf {\small Технологии/Фреймворки:} & \textmd {\small Linux, Git, NLTK, HuggingFace, PySpark, NumPy, Pandas, Scikit-learn, Docker, FastAPI, BeautifulSoup, OpenGL, Yolo, GreenPlum}


\end{document}
