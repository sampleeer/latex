\documentclass{article}
\usepackage{graphicx} % Required for inserting images
\usepackage[T2A]{fontenc}
\usepackage{amsmath}

\usepackage[utf8]{inputenc}
\usepackage[margin=2cm]{geometry} 

\begin{document}


\begin{center}
\textbf{\S \; 32.6.  ПРЕДЕЛЬНЫЕ  ВЕРОЯТНОСТИ  СОСТОЯНИЙ}\end{center}
 \\ 

\indent Уравнения Колмогорова дают возможность найти все вероятности состояний  как функции  времени.  Особый  интерес  представляет случай  при $t \to \infty $.  В этом случае получают \textit{предельные  (или  финальные) вероятности состояний}. Существование предельных вероятностей означает, что с течением времени в системе наступает \textit{стационарный режим:} она случайным образом меняет свои состояния, но вероятность $p_i$ каждого из них уже не зависит от времени. \\ 
\indent \textit{Предельная  вероятность} $p_i$ — это среднее относительное время пребывания системы в состоянии $S_i$ (сколько процентов времени система проводит в состоянии $S_i$).  
 Так  как предельные  вероятности $p_i$  не  зависят от времени, то  $\frac{d p_i(t)}{dt} = 0 \quad (i = 0, \ldots, n)$ 
, то есть в левой части уравнений Колмогорова получаем 0. Для решения этой системы удобно отрицательные слагаемые перенести из правой части исходной системы в левую. \\ \indent
При составлении $i-$го  уравнения системы уравнений Колмогорова для предельных вероятностей нужно рассмотреть состояние $S_i$, на размеченном графе состояний. В левой части уравнения указана сумма интенсив- \\ ностей всех выходящих из $S_i$ стрелок, умноженная на предельную вероятность $p_i$ Каждой входящей в $S_i$, из $S_j$ стрелке (на ней указана интенсивность $\lambda_{ji}$) в правой части уравнения соответствует слагаемое $\lambda_{ji}p_j$.  \\ В полученной системе уравнений одно уравнение будет выражаться через другие. \\ \indent
Поэтому какое-то одно уравнение этой системы нужно заменить уравнением
$p_0 + p_1 + ... + p_n = \sum\limits_{i=0}^{n} p_i = 1.$

\indent \; \; \; \textbf{Пример  98.} Найдем предельные вероятности для следующей системы.

\begin{figure}[h] 
\centering
\includegraphics{something 2024-05-03 в 16.22.53.png} % Путь к вашему изображению
\label{fig:my_label}
\end{figure}
\indent \; \; \;  Составим уравнения Колмогорова.
\\ 
\indent Из состояния $S_0$ выходят стрелки с интенсивностями 2 и 1. Поэтому в левой части соответствующего уравнения Колмогорова будет (2 + 1)$p_0$. В состояние $S_0$ входят стрелка с интенсивностью 3 из состояния $S_i$  (ей соответствует слагаемое $3p_1$,  в правой части уравнения Колмогорова) и стрелка с интенсивностью 4 из состояния $S_2$ (ей соответствует слагаемое $4p_2$ в правой части уравнения Колмогорова). \\ Получаем уравнение (2 + 1)$p_0$ = $3p_1 + 4p_2.$
 \\ \indent Из состояния $S_1$ выходят стрелки с интенсивностями 3 и 6. Поэтому в левой части соответствующего уравнения Колмогорова будет (3 + 6)$p_1$. В состояние $S_1$, входят стрелка с интенсивностью 1 из состояния $S_0$ (ей соответствует слагаемое $1 \times p_0$ в правой части уравнения Колмогорова) и стрелка с интенсивностью 5 из состояния $S_2$ (ей соответствует слагаемое $5p_2$ в правой части уравнения Колмогорова). \\  Получаем уравнение (3 + 6)$p_1$ = 1$p_0$ + 5$p_2$. И т. д. \\ \indent
 Система уравнений Колмогорова: \\ 
\[
\begin{aligned}
    &\left\{
    \begin{aligned}
       &(2 + 1)p_0 = 3p_1 + 4p_2,\\
        &(3 + 6)p_1 = 1p_0 + 5p_2, \\
        &(4 + 5)p_2 = 2p_0 + 6p_1,
    \end{aligned}
    \right.
    \quad \rightarrow \quad
    \left\{
    \begin{aligned}
        &3p_0 = 3p_1 + 4p_2, \\
        &9p_1 = 1p_0 + 5p_2, \\
        &9p_2 = 2p_0 + 6p_1,
    \end{aligned}
    \right.
\end{aligned}
\]
\indent Мы видим, что последнее уравнение есть сумма двух предыдущих уравнений.  Поэтому  вместо  него  включим  в  систему  уравнение
$p_0 + p_1 + p_2 = 1:$
\[
\begin{aligned}
    &\left\{
    \begin{aligned}
       &3p_0 = 3p_1 + 4p_2, \\
        &9p_1 = p_0 + 5p_2, \\
        &p_0 + p_1 + p_2 = 1.
    \end{aligned}
    \right.
    \quad \rightarrow \quad
    \left\{
    \begin{aligned}
         &3p_0 - 3p_1 - 4p_2 = 0, \\
        &9p_1 - p_0 - 5p_2 = 0, \\
        &p_2 = 1 - p_0 - p_1.
    \end{aligned}
    \right.
    \quad \rightarrow \quad
\end{aligned}
\]
\[
\begin{aligned}
    &\left\{
    \begin{aligned}
       &3p_0 - 3p_1 - 4(1 - p_0 - p_1) = 0, \\
        &9p_1 - p_0 - 5(1 - p_0 - p_1) = 0, \\
        &p_2 = 1 - p_0 - p_1.
    \end{aligned}
    \right.
    \quad \rightarrow \quad
    \left\{
    \begin{aligned}
         &7p_0 + p_1 = 4, \\
        &4p_0 - 14p_1 = 5, \\
        &p_2 = 1 - p_0 - p_1.
    \end{aligned}
    \right.
\end{aligned}
\]
\indent К первым двум уравнениям системы применим правило Крамера. Тогда:
\[
\begin{aligned}
&\triangle = 
\begin{array}{cc}
\left|\begin{array}{cc}
7 & 1 \\
4 & 14 \\
\end{array}\right|
\end{array} = 7 \times 14 - 1 \times 4 = 94; \\
&\triangle_{p_0} = 
\begin{array}{cc}
\left|\begin{array}{cc}
4 & 1 \\
5 & 14 \\
\end{array}\right|
\end{array} = 4 \times 14 - 1 \times 5 = 51; \\
&\triangle_{p_1} = 
\begin{array}{cc}
\left|\begin{array}{cc}
7 & 4 \\
4 & 5 \\
\end{array}\right|
\end{array} = 7 \times 5 - 4 \times 4 = 51;
\end{aligned}
\]
\indent 
$p_0 = \triangle_{p_0}/\triangle = 51/94 \approx 0,543;$ \\ 
\indent 
$p_1 = \triangle_{p_1}/\triangle = 19/94 \approx 0,202;$  \\ 
\indent 
$p_2 = 1 - p_0 - p_1 = 1 - 0,543 - 0,202 \approx 0,255.$ \\ \indent

Полученные результаты говорят о том, что в предельном стационарном режиме система в среднем 54.3\% времени будет находиться в состоянии $S_0$, 20.2\% времени будет находиться в состоянии $S_1$, и 25.5\% времени будет находиться в состоянии $S_2$.
\\ \indent
Если в состояниях $S_0, S_1$ и $S_2$ система приносит 10, 8 и 6 денежных единиц дохода соответственно, то средняя эффективность системы равна сумме произведений предельных вероятностей состояний и доходов в этих состояниях: $ 0,543 \times 10 + 0,202\times8 + 0,255\times6 = 8,576$ денежных единиц.
\\ \\
\indent \; \; \; \textbf{Задача 98.} Найти предельные вероятности для следующей системы.
\begin{figure}[h] 
\centering
\includegraphics[width=0.3\textwidth]{something 2024-05-03 в 18.46.03.png}
\label{fig:my_label}
\end{figure} \\ 
\indent Оценить среднюю эффективность системы, если в состояниях $S_0,  \; S_1$ и $S_2$ система приносит 11, 7 и 5 денежных единиц дохода соответственно.

\begin{center}
\textbf{\S \; 32.7.  ПРОЦЕСС  ГИБЕЛИ  И  РАЗМНОЖЕНИЯ}\end{center}
 \\ 
 
\indent В теории массового обслуживания широкое распространение получил  специальный  класс  случайных  процессов  — \textit{процесс  гибели и размножения}. Его размеченный граф состояний имеет следующий вид:


\begin{figure}[h] 
\centering
\includegraphics[width=0.7\textwidth]{something 2024-05-03 в 18.51.33.png}
\label{fig:my_label}
\end{figure} \\

Здесь $\lambda_{ij}$ — это интенсивность перехода из состояния $S_i$, в состояние $S_j$. Крайние состояния $S_0$ и $S_n$ имеют только одно соседнее состояние ($S_1$, и $S_{n-1}$ соответственно). Каждое из других состояний $S_i$, связано прямой связью с состоянием $S_{i+1}$ (верхняя стрелка с интенсивностью $\lambda_{i,j-1}$) и обратной связью с состоянием   (нижняя стрелка с интенсивностью $\lambda_{i, j-1}$), $i = 1, ..., n-1$.
\\  \indent
    В биологических задачах такие процессы описывают изменение численности особей в популяции. Предполагается, что все потоки простейшие.  Вместо запоминания общих  громоздких формул для предельных вероятностей $p_i$, можно очень просто найти $p_i$, из размеченного графа состояний для конкретной задачи.
\\ 
\begin{quote}
\textbf{Пример  99.} Найдем предельные вероятности для процесса гибели и размножения, размеченный граф состояний которого имеет следующий вид:
\end{quote}
\begin{figure}[h] 
\centering
\includegraphics[width=0.4\textwidth]{something 2024-05-03 в 20.21.18.png}
\label{fig:my_label}
\end{figure} \\


 Двигаемся по этому графу слева направо.  Вероятность $p_0$ — предельная вероятность состояния $S_0$. \\ 
\indent Следующее состояние  —  $S_1$. Оно связано с состоянием $S_0$ двумя стрелками с интенсивностями  1  и 6.  Пусть $p_1$  —  предельная вероятность состояния $S_1$. Имеем $p_1 = \frac{1}{6}p_0$ (интенсивность верхней стрелки пишем в числителе, интенсивность нижней стрелки пишем в знаменателе). \\ \indent
Следующее  состояние $S_2$.  Оно  связано  с  состоянием  $S_1$,  двумя стрелками с  интенсивностями  2  и  3.  Пусть $p_2$ —  предельная вероятность состояния $S_2$.  Имеем $p_2 = \frac{2}{3}p_1$, (интенсивность верхней стрелки пишем в числителе, нижней стрелки — в знаменателе) = $\frac{2}{3}\times\frac{1}{9}p_0 = \frac{5}{36}.$ \\ \indent

Следующее  состояние  $S_3$.  Оно  связано  с  состоянием $S_2$  двумя стрелками с интенсивностями 5  и 4.\\  Пусть $p_3$  — предельная  вероятность состояния $S_3$. Имеем $p_3 = \frac{5}{4}p_2$ (интенсивность верхней стрелки пишем в числителе, нижней стрелки — в знаменателе) $= \frac{5}{4}\times\frac{1}{9}p_0 = \frac{5}{36}p_0.$ \\ 
\\ \indent Так как $p_0 + p_1 + p_2 + p_3 = 1,$
то $1 = p_0 + \frac{1}{6}p_0 + \frac{1}{9}p_0 + \frac{5}{36}p_0 = \frac{51}{36}p_0.$ \\ 

\noindent Отсюда $p_0 = \frac{36}{51} \approx 0,706$
\\ \\  \indent Тогда $p_1 = \frac{1}{6}p_0 \approx
0,118, p_2 = \frac{1}{9}p_0 \approx 0,0078, p_3 = \frac{5}{36}p_0 \approx 0,098$.
\\ \\ 
\indent Эту же схему будем применять при анализе других СМО.

\begin{quote}
\textbf{Задача  99.} Найти предельные вероятности для процесса гибели и размножения, размеченный граф состояний которого имеет следующий вид:

\end{quote}
\begin{figure}[h] 
\centering
\includegraphics[width=0.5\textwidth]{something 2024-05-03 в 21.47.45.png}
\label{fig:my_label}
\end{figure} 

\vspace{5em} % Пространство в 1em

\begin{center}
\textbf{\S \; 32.8.  ОДНОКАНАЛЬНАЯ СМО С ОТКАЗАМИ}
\end{center}


СМО содержит один обслуживающий канал. На вход поступает простейший поток заявок с \\  интенсивностью $\lambda$. Образование очереди не допускается. Если заявка застала обслуживающий канал занятым, то она покидает систему.

\\ \indent
Время  обслуживания  заявки  есть  случайная  величина,  которая подчиняется  экспоненциальному  закону  распределения  с  параметром $\mu$. Среднее время обслуживания одной заявки $t_{\text{обсл}} =  1/\mu$.\\
\indent Возможные  состояния  СМО $S_0$  (канал  свободен)  и $S_1$ (канал занят). \\ \indent 
Нас  интересуют  следующие  показатели  эффективности  работы СМО: \\  
\indent 1)  абсолютная  пропускная способность $A$  (среднее число заявок, которое СМО может обслужить в единицу времени); \\ \indent 2) относительная  пропускная способность $Q$ (отношение среднего числа обслуживаемых в единицу времени заявок к среднему числу поступивших за это время заявок); \\ \indent 3)  вероятность отказа $p_{\text{отк}}$   (вероятность того,  что  заявка  покинет СМО необслуженной). \\ \\ \\ \indent Размеченный  граф  состояний  одноканальной  СМО  с  отказами имеет следующий вид:

\begin{figure}[h] 
\centering
\includegraphics[width=0.3\textwidth]{something 2024-05-03 в 22.10.36.png}
\label{fig:my_label}
\end{figure} \\ \\

\indent \; \; \; \textbf{Пример  100.} Одноканальная телефонная линия. Заявка-вызов, поступившая в момент, когда линия занята, получает отказ. Простей­ший  поток  заявок  поступает  с  интенсивностью $\lambda = 50 $ звонков/ч. Время обслуживания заявки есть случайная величина,  которая подчиняется  экспоненциальному закону распределения.  Средняя  продолжительность разговора $t_{\text{обсл}} = 3$  мин.  Определим  показатели  эффективности работы СМО. \\ \indent Данная телефонная линия — это одноканальная СМО с отказами. Время обслуживания $t_{\text{обсл}} = 3$ мин = 3/60 ч = 0,05 ч. Тогда интенсивность обслуживания $\mu = 1/t_{\text{обсл}} = 1/0,05 = 20$ звонков/ч. Размеченный граф состояний имеет следующий вид:
\begin{figure}[h] 
\centering
\includegraphics[width=0.27\textwidth]{something 2024-05-03 в 22.20.42.png}
\label{fig:my_label}
\end{figure} \\ 
\indent Пусть $p_0$  —  предельная  вероятность состояния $S_0$.  Состояние  $S_1$, связано с состоянием $S_0$ двумя стрелками с интенсивностями 50 и 20. \\ \\ \noindent
Пусть $p_1$  —  предельная  вероятность  состояния  $S_1$.  Тогда $p_1 = \frac{50}{20}p_0$ (интенсивность верхней стрелки пишем в числителе, интенсивность нижней стрелки пишем в знаменателе) $= 2,5p_0$. \\ \indent
Так как $p_0 + p_1 = 1,$ то  $1 = p_0 + 2,5p_0 = 3,5p_0$  Отсюда $p_0 = 1/3,5 \approx 0,286.$ Тогда $p_1 = 2,5p_0 \approx 0,714.$ \\ \indent

Вероятность отказа $p_{\text{отк}
}$ — это вероятность того, что линия занята, то есть предельная вероятность состояния $S_1$. Поэтому $p_{\text{отк}} = p_1 \approx 0,14.$ \\ \indent 
Относительная пропускная способность $Q = 1 - p_{\text{отк}} = 1 - 0,714 = 0,286$ Это вероятность того, что заявка будет обслужена. Абсолютная  пропускная  способность $A = \lambda Q = 50\times0,286 = 14,3$ звонка/ч, то есть в среднем в час СМО обслуживает 14,3 звонка. \\ \indent Мы видим,  что номинальная  пропускная способность телефонной линии $\mu \approx 20$ звонков/ч отличается от абсолютной пропускной способности $A  =  14,3$  звонка/ч  из-за  случайного  характера  потока звонков и случайности времени обслуживания. \\ \\
\indent \; \; \; \textbf{Задача  100.} Одноканальная телефонная линия. Заявка-вызов, поступившая в момент, когда линия занята, получает отказ. Простейший  поток  заявок  поступает  с  интенсивностью $\lambda = 60 $ звонков/ч. Время обслуживания заявки есть случайная величина, которая под­чиняется  экспоненциальному  закону  распределения.  Средняя  продолжительность разговора $t_{\text{обсл}} = 2,5$ мин. Определить показатели эффективности \\работы СМО.

\begin{center}
\textbf{\S \; 32.9.  МНОГОКАНАЛЬНАЯ  СМО С ОТКАЗАМИ (ЗАДАЧА ЭРЛАНГА)}
\end{center}



\indent СМО содержит $n$ обслуживающих каналов. На вход поступает простейший поток заявок с интенсивностью $\lambda$. Образование очереди не допускается. Если заявка застала все обслуживающие каналы занятыми, то она покидает систему. Если в момент поступления требвания имеется свободный канал, то он немедленно приступает к обслуживанию поступившего требования. Каждый канал может одновременно обслуживать только одно требование. Все каналы функционируют независимо.

\indent Время обслуживания заявки есть случайная величина, которая подчиняется экспоненциальному закону распределения с параметром $\mu$. Среднее время обслуживания одной заявки $t_{\text{обсл}} = 1/\mu$.




Возможные  состояния СМО $S_0$ (все каналы свободны), $S_1$,  (один канал занят, остальные свободны), $S_2$ (два канала заняты, остальные свободны),..., $S_n$ (все каналы заняты). \\ \indent
Приведенная  интенсивность  потока  заявок  (интенсивность  нагрузки канала) $p = \lambda/\mu$. \\ \indent 
Нас  интересуют  следующие  показатели  эффективности  работы СМО: \\ 
\indent
1)     абсолютная пропускная способность $А$  (среднее число заявок, которое СМО может обслужить в единицу времени);
\\ \indent 
2) относительная пропускная способность $Q$ (отношение среднего числа обслуживаемых в единицу времени заявок к среднему числу поступивших за это время заявок);
\\ \indent
3)  вероятность отказа ротк  (вероятность того,  что заявка покинет СМО необслуженной);
\\ \indent 
4) $p_0$ (вероятность того, что все обслуживающие каналы свободны);
\\ \indent 
5) $p_k$ (вероятность того, что в системе $k$  требований);
\\ \indent 
6) среднее число свободных от обслуживания каналов $N_0$;
\\ \indent 
7) коэффициент простоя каналов $K_{\text{пр}}$;
\\ \indent 
8) среднее число занятых обслуживанием каналов $N_{\text{зан}}$; 
\\ \indent 
9) коэффициент загрузки каналов $K_{\text{зан}}$. 
\\ \indent 
Размеченный  граф  состояний  многоканальной  СМО  с  отказами имеет следующий вид:

\begin{figure}[h] 
\centering
\includegraphics[width=0.8\textwidth]{something 2024-05-08 в 16.02.01.png}
\label{fig:my_label}
\end{figure} 

\indent \; \; \; \textbf{Пример  101.} Трехканальная телефонная линия.  Заявка-вызов, поступившая в момент, когда все $n = 3$ канала заняты, получает отказ. Простейший поток заявок поступает с интенсивностью \\ $\lambda = 60 $ звонков/ч. Время обслуживания заявки есть случайная величина, которая подчиняется  экспоненциальному  закону  распределения.  Средняя продолжительность разговора \; $t_{\text{обсл}} = 3$  мин.  Определим  показатели эффективности работы СМО. \\ 
\indent 
Данная телефонная линия — это многоканальная СМО с отказами $t_{\text{обсл}} = 3$ мин = 3/60 ч = 0,05 ч. Тогда интенсивность обслуживания $\mu = 1/t_{\text{обсл}} = 1 / 0,05 = 20$ звонков/ч. Приведенная интенсивность потока заявок $p = \lambda/\mu= 60/20 = 3$.  Размеченный граф состояний имеет следующий вид:


\begin{figure}[h] 
\centering
\includegraphics[width=0.5\textwidth]{something 2024-05-08 в 16.09.53.png}
\label{fig:my_label}
\end{figure} 

\indent $p_0$ — предельная вероятность состояния $S_0$. Состояние $S_1$, связано с состоянием $S_0$ двумя стрелками с интенсивностями $\lambda$ и $\mu$. \\ 
\indent Пусть $p_1$ — предельная вероятность состояния $S_1$. Имеем $p_1 = \frac{\lambda}{\mu}p_0$  (интенсивность верхней стрелки пишем в числителе, интенсивность нижней  стрелки  пишем  в  знаменателе) $= \rho p_0 = 3p_0.$ \\ Аналогично $p_2 = \frac{\lambda}{2\mu}p_1 = \frac{\rho}{2}p_1 = \frac{3}{2}\times 3p_0 = 4,5p_0; \; p_3 = \frac{\lambda}{3\mu}p_2=\frac{\rho}{3}p_2 = \frac{3}{3}\times 4,5p_0 = 4,5p_0.$
\\ \indent Так как $p_0 + p_1 + p_2 + p_3 = 1,$ то $1 = p_0 + 3p_0 + 4,5p_0 + 4,5p_0 = 13p_0.$ \\
Отсюда $p_0 = 1 /13 \approx 0,077$ (вероятность того, что все обслуживающие каналы свободны). \\ Тогда $p_1 = 3p_0 = 0,231, p2 = 4,5p_0 \approx 0,346, p3 = 4,5p_0 \approx 0,346.$.  Мы  нашли  вероятности  того,  что  в  системе $к$ требований, $k = 0, 1, 2, 3.$  \\ \indent

Вероятность отказа $p_{\text{отк}}$ — это вероятность того, что все каналы заняты,  то  есть  предельная  вероятность состояния  $S_3$.  Поэтому $p_{\text{отк}} = 3 = p_3 \approx 0,346$. \\ 
\indent
Относительная пропускная способность $Q = 1 - p_{\text{отк}} = 1 - 0,346 = 0,654$. Это вероятность того, что заявка будет обслужена. \\ \indent 
Абсолютная  пропускная способность $А = \lambda Q = 60\times0,654 =  39,24$ звонка/ч, то есть в среднем в час СМО обслуживает 39,24 звонка. \\ \indent 
Среднее число свободных от обслуживания каналов $N_0$ есть математическое ожидание числа свободных  каналов,  то есть число свободных каналов в каждом состоянии  надо умножить на предельную вероятность  этого  состояния  и  полученные  произведения  сложить: $N_0 = 3\times p_0 + 2\times p_1 + 1 \times p_2 + 0 \times p_3 = 3 \times 0,077 + 2 \times 0,231 + 1 \times 0,346 + 0\times 0,346 = 1,039 $. \\ \indent
Коэффициент простоя каналов $K_{\text{пр}} = N_0/n = 1,039/3 \approx 0,346$. \\ \indent
Среднее  число  занятых  обслуживанием  каналов $N_{\text{зан}} = А/\lambda = \lambda Q/\mu = \rho Q = 3\times0,654 = 1,962$. Мы видим, что из-за ошибок округления $n = N_0 + N_{\text{зан}} = 1,039 + 1,962 = 3,001$. \\ \indent 
Коэффициент загрузки каналов $K_{\text{зан}} = N_{\text{зан}}/n = 1,962/3 = 0,654.$
\\ \\ 
\indent \textbf{Задача 101.} 
Трехканальная телефонная линия.  Заявка-вызов, поступившая в момент, когда все $n = 3$ канала заняты, получает отказ. Простейший поток заявок поступает с интенсивностью $\lambda = 50$ звонков/ч. Время обслуживания заявки есть случайная величина, которая подчиняется  экспоненциальному  закону  распределения.  Средняя продолжительность разговора $t_{\text{обсл}} = 2,5$ мин. Определить показатели эффективности работы СМО. \\ \\ 
\noindent
Для  снижения  вероятности отказа  нужно увеличить число  каналов обслуживания. 
\\ \indent
Замечание.  Предельную вероятность состояния $S_k$ можно определить по следующей формуле: \\  $p_k = \frac{\rho^k}{k!}p_0, \; k=0, 1, ..., n$. При больших $n$ можно воспользоваться приближенной формулой: 
\[
p_k \approx \frac{\text{Ф}\left(\frac{k + 0,5 - p}{\sqrt{p}} \right) - \text{Ф}\left(\frac{k - 0,5 - p}{\sqrt{p}} \right)}{0,5 + \text{Ф}\left(\frac{k + 0,5 - p}{\sqrt{p}} \right)},
\]
где $\text{Ф(x)} = \frac{1}{\sqrt{2\pi}} \int\limits_0^x e^{-\frac{t^2}{2}}dt$ - функция Лапласа. \\ 
\indent Мастер функций $f_x$ пакета  Excel  позволяет  вычислить функцию Лапласа:\\
$\text{Ф(x)} = \text{НОРМРАСП} \; (x; 0; 1; 1) - 0,5.$



\begin{center}
\textbf{\S \; 32.10.  ОДНОКАНАЛЬНАЯ СМО С НЕОГРАНИЧЕННОЙ ОЧЕРЕДЬЮ}
\end{center}


 
\indent СМО содержит один обслуживающий канал.  На вход поступает простейший поток заявок\\  с интенсивностью $\lambda$.  Если заявка застала обслуживающий  канал занятым, то она встает в очередь и ожидает начала обслуживания. \\ 
\indent Время  обслуживания  заявки  есть  случайная  величина,  которая подчиняется  экспоненциальному закону распределения  с  параметром $\mu$. Среднее время обслуживания одной заявки $t_{\text{обсл}} = 1/\mu$. \\ \indent
Возможные состояния СМО $S_0$ (канал свободен), $S_1$, (канал занят, очереди нет), $S_2$ (канал занят, в очереди одна заявка), $S_3$ (канал занят, в очереди две заявки) и т. д. Размеченный  граф состояний одноканальной  СМО с неограниченной очередью имеет следующий вид:
\begin{figure}[h] 
\centering
\includegraphics[width=0.6\textwidth]{something 2024-05-10 в 19.54.46.png}
\label{fig:my_label}
\end{figure} 

При $\rho = \lambda \mu < 1$ существуют предельные вероятности. Найдем их. $p_0 -$ предельная вероятность состояния $S_0$. Состояние $S_1$, связано с состоянием $S_0$ двумя стрелками с интенсивностями $\lambda$ и $\mu$. \\ \indent
Пусть $p_1$  — предельная вероятность состояния $S_1$.  Как и раньше $p_1 = \frac{\lambda}{\mu}p_0$ (интенсивность верхней стрелки пишем в числителе, интенсивность нижней стрелки пишем в знаменателе) $= \rho р_0$. Аналогично  $p_2 = \frac{\lambda}{\mu}p_1 = \rho \times\rho p_0 = \rho ^2 p_0; \; p_3 = \frac{\lambda}{\mu}p_2= \rho \times \rho^2 p_0 = \rho^3 p_0.$ И т.д. $p_k = \rho^k p_0, \; k = 0, 1, 2, ... $ \\ 
\indent Так как $p_0 + p_1 + p_2 + p_3 + ... = 1 \; \text{,то} \; 1= p_0 + \rho p_0 + \rho ^2 p_0 + \rho^3 p_0 + ... = p_0(1+\rho + \rho ^2 +\rho^3 + ...).$
В скобках указана сумма бесконечно убывающей  геометрической  прогрессии  с  первым  элементом $b_1  = 1$
и  знаменателем $q = \rho < 1$. \;  Эта  сумма  равна $\frac{b_1}{1 - q} = \frac{1}{1 - \rho}$.  Тогда $1 = p_0\times\frac{1}{1-\rho}$, то есть $p_0 = 1 - \rho$. Это вероятность того, что канал свободен.
\\ \indent Тогда вероятность состояния $S_k$ (канал занят, в очереди $k - 1$ заявка) равна $p_k= \rho^k р_0 = \rho^k (1 -  \rho), \; к = 1, 2, 3,...$ \\ \indent 
Вероятность того, что канал занят, равна $p_{\text{зан}} = 1 - p_0 = 1 - (1 - \rho) = \rho$. 
\\ \\ \\ \\ \\ \indent Найдем среднее число заявок в системе $L_{\text{сист}}$ по формуле математического ожидания: число требований в каждом состоянии $S_k$ надо умножить на предельную вероятность этого состояния и полученные произведения суммировать. 
\begin{multline*}
        L_{\text{сист}} = 0\times p_0 + 1 \times p_1 + 2 \times p_2 + 3\times p_3 + ... = 1\times \rho(1-\rho) - 2 \times\rho ^ 2 (1-\rho) + 3 \times \rho^3 (1-\rho) + ... = \rho(1-\rho)(1+2\rho + 3\rho^2+...)= \\ = \rho(1-\rho)\left( \frac{d\rho}{d\rho} + \frac{d\rho^2}{d\rho} + \frac{d\rho^3}{d\rho} + ... \right) \rho(1-\rho)\frac{d}{d\rho}(\rho + rho^2 + \rho^3 + ...).       
\end{multline*}
\indent
Мы предполагаем, что выполнены условия, при которых возможно  поменять местами операции дифференцирования  и суммирования. \\ \indent 
В скобках указана сумма бесконечно убывающей геометрической прогрессии  с  первым  элементом $b_1  =  \rho$  и  знаменателем $q =  \rho  < 1.$
\\ 
\noindent Эта сумма равна $\frac{b_1}{1 - q} = \frac{\rho}{1 - \rho}.$ \\ Отсюда \[ L_{\text{сист}} =  \rho (1-\rho)\frac{d}{d\rho}\left(\frac{\rho}{1 - \rho} \right) = \rho(1 - \rho) \frac{\frac{d\rho}{d\rho} (1-\rho) - \rho \frac{d(1-\rho)}{d\rho}}{(1-\rho)^2} = \rho(1-\rho)\frac{1 - \rho + \rho}{(1-\rho)^2} = \frac{\rho}{1 - \rho}.
\]

\indent Тогда среднее время  пребывания заявки  в системе  $T_{\text{сист}} = L_{\text{сист}}/\lambda$  (формула Литтла). В  $T_{\text{сист}}$ входят время обслуживания заявки и время в очереди. \\ \indent 
Найдем $L_{\text{обсл}}$ среднее число заявок, находящихся под обслуживанием. \\ Канал либо свободен с вероятностью $р_0$, либо занят с вероятностью  $1- p_0$.  \\ Поэтому $L_{\text{обсл}} = 0\times p_0 + 1 \times(1-p_0)=1-p_0 = \rho.$ \\Тогда среднее число заявок в очереди $L_{\text{оч}} = L_{\text{сист}} - L_{\text{обсл}} = \frac{\rho}{1 - \rho} - \rho = \frac{\rho^2}{1-\rho}.$
Отсюда среднее время пребывания заявки в очереди $T_{\text{оч}} = L_{\text{оч}}/\lambda$ (этот результат также называют формулой Литтла).
\\ \\ 
\indent \textbf{Пример  102.} 
 Магазин с одним продавцом. Предполагается, что простейший  поток покупателей поступает с интенсивностью $\lambda = 20$ человек/ч. Время обслуживания заявки — случайная величина, которая подчиняется экспоненциальному закону распределения с параметром $\mu = 25$ человек/ч. Определим: \\ \indent
1) среднее время пребывания покупателя в очереди; \\ \indent 2) среднюю длину очереди;\\ \indent 3) среднее число покупателей в магазине; \\ \indent 4) среднее время пребывания покупателя в магазине; \\ \indent 5) вероятность того, что в магазине не окажется покупателей; \\ \indent 6) вероятность того, что в магазине окажется ровно 4 покупателя. \\ \indent Данный магазин — одноканальная СМО с неограниченной очередью. Размеченный граф состояний имеет следующий вид:
\begin{figure}[h] 
\centering
\includegraphics[width=0.38\textwidth]{something 2024-05-10 в 23.50.58.png}
\label{fig:my_label}
\end{figure} 
\\
\indent Так как $\rho = \lambda/\mu = 20/25 = 0,8 < 1$, 
 то существуют предельные вероятности. \\ \noindent Вероятность того, что в магазине не окажется покупателей, равна $p_0 = 1 - \rho = 1 - 0,8 = 0,2$. \\ \noindent Вероятность того,  что в магазине окажется ровно 4 покупателя, равна $p_4 = \rho^4 p_0 = 0,8^4\times0,2 \approx 0,082.$
 \\ \noindent Средняя длина очереди $L_{\text{оч}} = \frac{\rho^2}{1 - \rho} = \frac{0,8^2}{1 - 0,8} = 3,2.$ 
 \\ \noindent Среднее  время  пребывания  покупателя  в очереди: \\
 $T_{\text{оч}} = L_{\text{оч}} / \lambda =3,2/20=0,16 \; \text{ч} = 0,16\times60 \; \text{мин} = 9,6 \;  \text{мин.}$ \\ \noindent
 Среднее число покупателей в магазине $L_{\text{сист}} = \frac{\rho}{1-\rho} = \frac{0,8}{1 - 0,8} = 4.$ \\ \noindent
Среднее   время   пребывания   покупателя   в   магазине $T_{\text{сист}} = L_{\text{сист}}/\lambda = 4/20 = 0,2 \; \text{ч} = 0,2 \times 60 \text{мин} = 12 \text{мин.}$
\\ \\  \\ \\  \\ \\  
\indent \textbf{Задача  102.}  Магазин с одним продавцом.  Предполагается, что простейший поток покупателей поступает с интенсивностью $\lambda = 10$ человек/ч. Время обслуживания заявки есть случайная величина, которая  подчиняется экспоненциальному закону распределения с параметром $\mu = 15$ человек/ч. Определить: \\ \indent 1) среднее время пребывания покупателя в очереди; \\ \indent  2) среднюю длину очереди; \\ \indent 3) среднее число покупателей в магазине; \\ \indent 4) среднее время пребывания покупателя в магазине;\\ \indent  5) вероятность того, что в магазине не окажется покупателей; \\ \indent  6) вероятность того, что в магазине окажется ровно 4 покупателя.


\\ \\ \\ \\

\begin{center}
    \textbf{Решение задач:}
\end{center} \\  
\indent \; \; \; \textbf{Задача 98.} Найти предельные вероятности для следующей системы.
\begin{figure}[h] 
\centering
\includegraphics[width=0.3\textwidth]{something 2024-05-03 в 18.46.03.png}
\label{fig:my_label}
\end{figure} \\ 
\indent Оценить среднюю эффективность системы, если в состояниях $S_0,  \; S_1$ и $S_2$ система приносит 11, 7 и 5 денежных единиц дохода соответственно. \\ 

\indent Из состояния $S_0$ выходят стрелки с интенсивностями 3 и 2. Поэтому в левой части соответствующего уравнения Колмогорова будет (2 + 3)$p_0$. В состояние $S_0$ входят стрелка с интенсивностью 5 из состояния $S_1$  (ей соответствует слагаемое $5p_1$,  в правой части уравнения Колмогорова) и стрелка с интенсивностью 6 из состояния $S_2$ (ей соответствует слагаемое $6p_2$ в правой части уравнения Колмогорова). \\ Получаем уравнение (2 + 3)$p_0$ = $5p_1 + 6p_2.$
 \\ \indent Из состояния $S_1$ выходят стрелки с интенсивностями 5 и 4. Поэтому в левой части соответствующего уравнения Колмогорова будет (5 + 4)$p_1$. В состояние $S_1$, входят стрелка с интенсивностью 2 из состояния $S_0$ (ей соответствует слагаемое $2 \times p_0$ в правой части уравненияКолмогорова) и стрелка с интенсивностью 7 из состояния $S_2$ (ей соответствует слагаемое $7p_2$ в правой части уравнения Колмогорова). \\  Получаем уравнение (5+4)$p_1$ = 2$p_0$ + 7$p_2$. И т. д. \\ \indent
 Система уравнений Колмогорова: \\ 
\[
\begin{aligned}
    &\left\{
    \begin{aligned}
       &(2 + 1)p_0 = 3p_1 + 4p_2,\\
        &(3 + 6)p_1 = 1p_0 + 5p_2, \\
        &(4 + 5)p_2 = 2p_0 + 6p_1,
    \end{aligned}
    \right.
    \quad \rightarrow \quad
    \left\{
    \begin{aligned}
        &3p_0 = 3p_1 + 4p_2, \\
        &9p_1 = 1p_0 + 5p_2, \\
        &9p_2 = 2p_0 + 6p_1,
    \end{aligned}
    \right.
\end{aligned}
\]
\indent Мы видим, что последнее уравнение есть сумма двух предыдущих уравнений.  Поэтому  вместо  него  включим  в  систему  уравнение
$p_0 + p_1 + p_2 = 1:$
\[
\begin{aligned}
    &\left\{
    \begin{aligned}
       &3p_0 = 3p_1 + 4p_2, \\
        &9p_1 = p_0 + 5p_2, \\
        &p_0 + p_1 + p_2 = 1.
    \end{aligned}
    \right.
    \quad \rightarrow \quad
    \left\{
    \begin{aligned}
         &3p_0 - 3p_1 - 4p_2 = 0, \\
        &9p_1 - p_0 - 5p_2 = 0, \\
        &p_2 = 1 - p_0 - p_1.
    \end{aligned}
    \right.
    \quad \rightarrow \quad
\end{aligned}
\]
\[
\begin{aligned}
    &\left\{
    \begin{aligned}
       &3p_0 - 3p_1 - 4(1 - p_0 - p_1) = 0, \\
        &9p_1 - p_0 - 5(1 - p_0 - p_1) = 0, \\
        &p_2 = 1 - p_0 - p_1.
    \end{aligned}
    \right.
    \quad \rightarrow \quad
    \left\{
    \begin{aligned}
         &7p_0 + p_1 = 4, \\
        &4p_0 - 14p_1 = 5, \\
        &p_2 = 1 - p_0 - p_1.
    \end{aligned}
    \right.
\end{aligned}
\]
\indent К первым двум уравнениям системы применим правило Крамера. Тогда:
\[
\begin{aligned}
&\triangle = 
\begin{array}{cc}
\left|\begin{array}{cc}
7 & 1 \\
4 & 14 \\
\end{array}\right|
\end{array} = 7 \times 14 - 1 \times 4 = 94; \\
&\triangle_{p_0} = 
\begin{array}{cc}
\left|\begin{array}{cc}
4 & 1 \\
5 & 14 \\
\end{array}\right|
\end{array} = 4 \times 14 - 1 \times 5 = 51; \\
&\triangle_{p_1} = 
\begin{array}{cc}
\left|\begin{array}{cc}
7 & 4 \\
4 & 5 \\
\end{array}\right|
\end{array} = 7 \times 5 - 4 \times 4 = 51;
\end{aligned}
\]
\indent 
$p_0 = \triangle_{p_0}/\triangle = 51/94 \approx 0,543;$ \\ 
\indent 
$p_1 = \triangle_{p_1}/\triangle = 19/94 \approx 0,202;$  \\ 
\indent 
$p_2 = 1 - p_0 - p_1 = 1 - 0,543 - 0,202 \approx 0,255.$ \\ \indent

Полученные результаты говорят о том, что в предельном стационарном режиме система в среднем 54.3\% времени будет находиться в состоянии $S_0$, 20.2\% времени будет находиться в состоянии $S_1$, и 25.5\% времени будет находиться в состоянии $S_2$.
\\ \indent
Если в состояниях $S_0, S_1$ и $S_2$ система приносит 10, 8 и 6 денежныхединиц дохода соответственно, то средняя эффективность системыравна сумме произведений предельных вероятностей состояний и доходов в этих состояниях: $ 0,543 \times 10 + 0,202\times8 + 0,255\times6 = 8,576$ денежных единиц.
\\ \\
\begin{quote}
\textbf{Задача  99.} Найти предельные вероятности для процесса гибели и размножения, размеченный граф состояний которого имеет следующий вид:

\end{quote}
\begin{figure}[h] 
\centering
\includegraphics[width=0.5\textwidth]{something 2024-05-03 в 21.47.45.png}
\label{fig:my_label}
\end{figure} 

\vspace{5em} % Пространство в 1em


 Двигаемся по этому графу слева направо.  Вероятность $p_0$ — предельная вероятность состояния $S_0$. \\ 
\indent Следующее состояние  —  $S_1$. Оно связано с состоянием $S_0$ двумя стрелками с интенсивностями  3  и 5.  Пусть $p_1$  —  предельная вероятность состояния $S_1$. Имеем $p_1 = \frac{5}{3}p_0$ (интенсивность верхней стрелки пишем в числителе, интенсивность нижней стрелки пишем в знаменателе). \\ \indent
Следующее  состояние $S_2$.  Оно  связано  с  состоянием  $S_1$,  двумя стрелками с  интенсивностями  1  и  2.  Пусть $p_2$ —  предельная вероятность состояния $S_2$.  Имеем $p_2 = \frac{1}{2}p_1$, (интенсивность верхней стрелки пишем в числителе, нижней стрелки — в знаменателе) = $\frac{5}{3}\times\frac{1}{2}p_0 = \frac{5}{6}.$ \\ \indent

Следующее  состояние  $S_3$.  Оно  связано  с  состоянием $S_2$  двумя стрелками с интенсивностями 6  и 4.\\  Пусть $p_3$  — предельная  вероятность состояния $S_3$. Имеем $p_3 = \frac{4}{6}p_2$ (интенсивность верхней стрелки пишем в числителе, нижней стрелки — в знаменателе) $= \frac{4}{6}\times\frac{5}{6}p_0 = \frac{20}{36}p_0.$ \\ 
\\ \indent Так как $p_0 + p_1 + p_2 + p_3 = 1,$
то $1 = p_0 + \frac{5}{3}p_0 + \frac{5}{6}p_0 + \frac{20}{36}p_0 = \frac{110}{36}p_0.$ \\ 

\noindent Отсюда $p_0 = \frac{36}{110} \approx 0,254$
\\ \\  \indent Тогда $p_1 = \frac{3}{5}p_0 \approx
0,423, p_2 = \frac{5}{6}p_0 \approx 0,211, p_3 = \frac{20}{36}p_0 \approx 0,112$.
\\ \\ 
\\ \\ \\
\textbf{Задача  100.} Одноканальная телефонная линия. Заявка-вызов, поступившая в момент, когда линия занята, получает отказ. Простейший  поток  заявок  поступает  с  интенсивностью $\lambda = 60 $ звонков/ч. Время обслуживания заявки есть случайная величина, которая подчиняется  экспоненциальному  закону  распределения.  Средняя  продолжительность разговора $t_{\text{обсл}} = 2,5$ мин. Определить показатели эффективности \\работы СМО. \\ \indent Данная телефонная линия — это одноканальная СМО с отказами. Время обслуживания $t_{\text{обсл}} = 2.5$ мин = 2.5/60 ч = 0,04 ч. Тогда интенсивность обслуживания $\mu = 1/t_{\text{обсл}} = 1/0,04 = 24$ звонков/ч. Размеченный граф состояний имеет следующий вид:
\\
\indent Пусть $p_0$  —  предельная  вероятность состояния $S_0$.  Состояние  $S_1$, связано с состоянием $S_0$ двумя стрелками с интенсивностями 60 и 24. \\ \\ \noindent
Пусть $p_1$  —  предельная  вероятность  состояния  $S_1$.  Тогда $p_1 = \frac{60}{24}p_0$ (интенсивность верхней стрелки пишем в числителе, интенсивность нижней стрелки пишем в знаменателе) $= 2,5p_0$. \\ \indent
Так как $p_0 + p_1 = 1,$ то  $1 = p_0 + 2,5p_0 = 3,5p_0$  Отсюда $p_0 = 1/3,5 \approx 0,286.$ Тогда $p_1 = 2,4p_0 \approx 0,714.$ \\ \indent

Вероятность отказа $p_{\text{отк}
}$ — это вероятность того, что линия занята, то есть предельная вероятность состояния $S_1$. Поэтому $p_{\text{отк}} = p_1 \approx 0,14.$ \\ \indent 
Относительная пропускная способность $Q = 1 - p_{\text{отк}} = 1 - 0,714 = 0,286$ Это вероятность того, что заявка будет обслужена. Абсолютная  пропускная  способность $A = \lambda Q = 60\times0,286 = 17,16$ звонка/ч, то есть в среднем в час СМО обслуживает 17,16 звонка. \\ \indent Мы видим,  что номинальная  пропускная способность телефонной линии $\mu \approx 25$ звонков/ч отличается от абсолютной пропускной способности $A  =  17,16$  звонка/ч  из-за  случайного  характера  потока звонков и случайности времени обслуживания. \\ \\
\indent \; \; \; 
\\ \\ 
\indent \textbf{Задача 101.} 
Трехканальная телефонная линия.  Заявка-вызов, поступившая в момент, когда все $n = 3$ канала заняты, получает отказ. Простейший поток заявок поступает с интенсивностью $\lambda = 50$ звонков/ч. Время обслуживания заявки есть случайная величина, которая подчиняется  экспоненциальному  закону  распределения.  Средняя продолжительность разговора $t_{\text{обсл}} = 2,5$ мин. Определить показатели эффективности работы СМО. \\ \\ 
\noindent
Для  снижения  вероятности отказа  нужно увеличить число  каналов обслуживания. 
\\ \indent
Замечание.  Предельную вероятность состояния $S_k$ можно определить по следующей формуле: \\  $p_k = \frac{\rho^k}{k!}p_0, \; k=0, 1, ..., n$. При больших $n$ можно воспользоваться приближенной формулой: 
\[
p_k \approx \frac{\text{Ф}\left(\frac{k + 0,5 - p}{\sqrt{p}} \right) - \text{Ф}\left(\frac{k - 0,5 - p}{\sqrt{p}} \right)}{0,5 + \text{Ф}\left(\frac{k + 0,5 - p}{\sqrt{p}} \right)},
\]
где $\text{Ф(x)} = \frac{1}{\sqrt{2\pi}} \int\limits_0^x e^{-\frac{t^2}{2}}dt$ - функция Лапласа. \\ 
\indent Мастер функций $f_x$ пакета  Excel  позволяет  вычислить функцию Лапласа:\\
$\text{Ф(x)} = \text{НОРМРАСП} \; (x; 0; 1; 1) - 0,5.$

\indent $p_0$ — предельная вероятность состояния $S_0$. Состояние $S_1$, связано с состоянием $S_0$ двумя стрелками с интенсивностями $\lambda$ и $\mu$. \\ 
\indent Пусть $p_1$ — предельная вероятность состояния $S_1$. Имеем $p_1 = \frac{\lambda}{\mu}p_0$  (интенсивность верхней стрелки пишем в числителе, интенсивность нижней  стрелки  пишем  в  знаменателе) $= \rho p_0 = 3p_0.$ \\ Аналогично $p_2 = \frac{\lambda}{2\mu}p_1 = \frac{\rho}{2}p_1 = \frac{3}{2}\times 3p_0 = 4,5p_0; \; p_3 = \frac{\lambda}{3\mu}p_2=\frac{\rho}{3}p_2 = \frac{3}{3}\times 4,5p_0 = 4,5p_0.$
\\ \indent Так как $p_0 + p_1 + p_2 + p_3 = 1,$ то $1 = p_0 + 3p_0 + 4,5p_0 + 4,5p_0 = 13p_0.$ \\
Отсюда $p_0 = 1 /13 \approx 0,077$ (вероятность того, что все обслуживающие каналы свободны). \\ Тогда $p_1 = 3p_0 = 0,231, p2 = 4,5p_0 \approx 0,346, p3 = 4,5p_0 \approx 0,346.$.  Мы  нашли  вероятности  того,  что  в  системе $к$ требований, $k = 0, 1, 2, 3.$  \\ \indent

Вероятность отказа $p_{\text{отк}}$ — это вероятность того, что все каналы заняты,  то  есть  предельная  вероятность состояния  $S_3$.  Поэтому $p_{\text{отк}} = 3 = p_3 \approx 0,346$. \\ 
\indent
Относительная пропускная способность $Q = 1 - p_{\text{отк}} = 1 - 0,346 = 0,654$. Это вероятность того, что заявка будет обслужена. \\ \indent 
Абсолютная  пропускная способность $А = \lambda Q = 60\times0,654 =  39,24$ звонка/ч, то есть в среднем в час СМО обслуживает 39,24 звонка. \\ \indent 
Среднее число свободных от обслуживания каналов $N_0$ есть математическое ожидание числа свободных  каналов,  то есть число свободных каналов в каждом состоянии  надо умножить на предельную вероятность  этого  состояния  и  полученные  произведения  сложить: $N_0 = 3\times p_0 + 2\times p_1 + 1 \times p_2 + 0 \times p_3 = 3 \times 0,077 + 2 \times 0,231 + 1 \times 0,346 + 0\times 0,346 = 1,039 $. \\ \indent
Коэффициент простоя каналов $K_{\text{пр}} = N_0/n = 1,039/3 \approx 0,346$. \\ \indent
Среднее  число  занятых  обслуживанием  каналов $N_{\text{зан}} = А/\lambda = \lambda Q/\mu = \rho Q = 3\times0,654 = 1,962$. Мы видим, что из-за ошибок округления $n = N_0 + N_{\text{зан}} = 1,039 + 1,962 = 3,001$. \\ \indent 
Коэффициент загрузки каналов $K_{\text{зан}} = N_{\text{зан}}/n = 1,962/3 = 0,654.$
\\ \\ 
\\ \\ 
\textbf{Задача 102.} Магазин с одним продавцом. Предполагается, что простейший поток покупателей поступает с интенсивностью $\lambda = 10$ человек/ч. Время обслуживания заявки есть случайная величина, которая подчиняется экспоненциальному закону распределения с параметром $\mu = 15$ человек/ч. Определить:
\begin{enumerate}
    \item среднее время пребывания покупателя в очереди;
    \item среднюю длину очереди;
    \item среднее число покупателей в магазине;
    \item среднее время пребывания покупателя в магазине;
    \item вероятность того, что в магазине не окажется покупателей;
    \item вероятность того, что в магазине окажется ровно 3 покупателя.
\end{enumerate}

Так как $\rho = \frac{\lambda}{\mu} = \frac{10}{15} = \frac{2}{3} < 1$, то существуют предельные вероятности.

Вероятность того, что в магазине не окажется покупателей, равна $p_0 = 1 - \rho = 1 - \frac{2}{3} = \frac{1}{3} = 0,333$.

Вероятность того, что в магазине окажется ровно 3 покупателя, равна $p_3 = \rho^3 p_0 = \left(\frac{2}{3}\right)^3 \times \frac{1}{3} = \frac{8}{27} \times \frac{1}{3} \approx 0,099$.

Средняя длина очереди $L_{\text{оч}} = \frac{\rho^2}{1 - \rho} = \frac{\left(\frac{2}{3}\right)^2}{1 - \frac{2}{3}} = \frac{4}{3} = 1,333$.

Среднее время пребывания покупателя в очереди: 
0,888 \text{ мин}. 

Среднее число покупателей в магазине $L_{\text{сист}} = \frac{\rho}{1-\rho} = \frac{\frac{2}{3}}{1 - \frac{2}{3}} = 2$.

Среднее время пребывания покупателя в магазине: 
\[ T_{\text{сист}} = \frac{L_{\text{сист}}}{\lambda} = \frac{2}{10} = \frac{1}{5} \text{ ч} = 12 \text{ мин}. \]


\end{document}




\end{document}



